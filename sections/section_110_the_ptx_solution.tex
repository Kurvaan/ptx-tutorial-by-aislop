\subsubsection{The PTX Solution}

We developed a custom fused kernel that:
\begin{enumerate}
    \item Computes the query-key products directly into shared memory
    \item Computes the softmax normalization within each attention head
    \item Immediately applies the normalized weights to the value matrix
\end{enumerate}

Key optimizations in the PTX implementation included:
\begin{itemize}
    \item Carefully tiled matrix multiplications to maximize shared memory reuse
    \item Cooperative computation of softmax max and sum values using warp shuffles
    \item Asynchronous global memory loads to overlap computation and data movement
\end{itemize}

Here's a snippet showing the warp-cooperative max-finding for softmax:

\begin{lstlisting}[style=ptx]
// Find maximum value across a warp for softmax stability
mov.f32 %f_max, %f_val;           // Start with this thread's value

// Warp reduction to find maximum (log2(32) = 5 steps)
.reg .f32 %f_other;
.reg .pred %p_max;

// Shuffle step 1 (stride 16)
shfl.sync.down.b32 %f_other, %f_max, 16, 0x1f;
max.f32 %f_max, %f_max, %f_other;

// Shuffle step 2 (stride 8)
shfl.sync.down.b32 %f_other, %f_max, 8, 0x1f;
max.f32 %f_max, %f_max, %f_other;

// Shuffle step 3 (stride 4)
shfl.sync.down.b32 %f_other, %f_max, 4, 0x1f;
max.f32 %f_max, %f_max, %f_other;

// Shuffle step 4 (stride 2)
shfl.sync.down.b32 %f_other, %f_max, 2, 0x1f;
max.f32 %f_max, %f_max, %f_other;

// Shuffle step 5 (stride 1)
shfl.sync.down.b32 %f_other, %f_max, 1, 0x1f;
max.f32 %f_max, %f_max, %f_other;

// Broadcast max to all threads in the warp
shfl.sync.idx.b32 %f_max, %f_max, 0, 0x1f;
\end{lstlisting}

