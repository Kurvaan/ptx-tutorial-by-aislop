\subsection{Control Flow}

Control flow in PTX is implemented with predicates and branches, rather than structured statements:

\begin{lstlisting}[language=C++]
// CUDA C++
if (value > 0.0f) {
    result = value;
} else {
    result = 0.0f;
}
\end{lstlisting}

\begin{lstlisting}[style=ptx]
// PTX
setp.gt.f32 %p1, %f_value, 0f00000000;  // Compare value > 0.0
@%p1 mov.f32 %f_result, %f_value;       // If true, result = value
@!%p1 mov.f32 %f_result, 0f00000000;    // If false, result = 0.0
\end{lstlisting}

For more complex control flow, PTX uses explicit branches:

\begin{lstlisting}[language=C++]
// CUDA C++
if (value > 10.0f) {
    // Complex operation 1
} else {
    // Complex operation 2
}
\end{lstlisting}

\begin{lstlisting}[style=ptx]
// PTX
setp.gt.f32 %p1, %f_value, 0f41200000;  // Compare value > 10.0
@!%p1 bra ELSE_BRANCH;                  // If not true, branch to ELSE

// Complex operation 1
// ...

bra END_IF;                             // Skip the else block

ELSE_BRANCH:
// Complex operation 2
// ...

END_IF:
// Continue with code after if-else
\end{lstlisting}

Loops follow a similar pattern:

\begin{lstlisting}[language=C++]
// CUDA C++
for (int i = 0; i < 10; i++) {
    // Loop body
}
\end{lstlisting}

\begin{lstlisting}[style=ptx]
// PTX
mov.u32 %r_i, 0;                    // Initialize i = 0

LOOP_START:
setp.ge.u32 %p1, %r_i, 10;          // Check if i >= 10
@%p1 bra LOOP_END;                  // If true, exit loop

// Loop body
// ...

add.u32 %r_i, %r_i, 1;              // Increment i
bra LOOP_START;                      // Go back to start

LOOP_END:
// Continue with code after loop
\end{lstlisting}

