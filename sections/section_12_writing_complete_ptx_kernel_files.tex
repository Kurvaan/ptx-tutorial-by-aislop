\subsubsection{Writing Complete PTX Kernel Files}

For more extensive control, you can write entire kernels in PTX. This involves creating a \texttt{.ptx} file containing the complete kernel implementation:

\begin{lstlisting}[style=ptx]
.version 8.0
.target sm_80
.address_size 64

.visible .entry myKernel(
    .param .u64 param0,  // pointer parameter
    .param .u64 param1   // another pointer
) {
    // PTX instructions for the kernel body
    // ...
    ret;
}
\end{lstlisting}

A PTX source file typically starts with directives specifying:
\begin{itemize}
    \item The PTX version (\texttt{.version})
    \item Target architecture (\texttt{.target})
    \item Address size (\texttt{.address\_size})
\end{itemize}

Then it declares one or more kernel functions using \texttt{.visible .entry} (for kernels callable from host code) or \texttt{.func} (for device functions callable from other device code).

