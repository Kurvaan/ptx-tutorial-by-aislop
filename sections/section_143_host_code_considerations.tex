\subsubsection{Host Code Considerations}

When using PTX kernels that leverage Tensor Cores, the host code needs to ensure:

\begin{itemize}
    \item The input data is in the correct format (e.g., FP16 for this example)
    \item Memory is properly aligned for efficient access
    \item The grid and block dimensions are appropriate for the warp-centric execution model
\end{itemize}

\begin{lstlisting}[language=C++]
// Example function setting up and launching a Tensor Core kernel
void launchTensorCoreMatrixMultiply(
    half* d_A, half* d_B, half* d_C,
    int m, int k, int n,
    int ldA, int ldB, int ldC) {
    
    // Ensure we're on a compatible device
    int deviceId;
    cudaGetDevice(&deviceId);
    cudaDeviceProp props;
    cudaGetDeviceProperties(&props, deviceId);
    
    if (props.major < 7) {
        throw std::runtime_error("Tensor Cores require compute capability 7.0+");
    }
    
    // Initialize CUDA Driver API
    cuInit(0);
    
    // Load the PTX module
    CUmodule module;
    cuModuleLoad(&module, "tensorMM.ptx");
    
    // Get function handle
    CUfunction tensorMM;
    cuModuleGetFunction(&tensorMM, module, "tensorMM");
    
    // Set up launch configuration
    // For Tensor Core operations, we typically organize by warps
    // Each warp handles a portion of the matrix
    dim3 blockDim(32, 1, 1);  // 1 warp per block
    dim3 gridDim(m/16, n/16, 1);  // Grid sized for 16x16 tiles
    
    // Set up kernel parameters
    void* args[] = {
        &d_A, &d_B, &d_C, &ldA, &ldB, &ldC
    };
    
    // Launch the kernel
    cuLaunchKernel(
        tensorMM,
        gridDim.x, gridDim.y, gridDim.z,
        blockDim.x, blockDim.y, blockDim.z,
        0, nullptr,
        args, nullptr
    );
    
    // Clean up
    cuModuleUnload(module);
}
\end{lstlisting}

