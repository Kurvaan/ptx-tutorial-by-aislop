\subsection{Final Thoughts}

Mastering PTX is not required for every CUDA developer, but it provides a powerful tool for performance-critical applications, especially in deep learning. Like assembly language in CPU programming, PTX gives you a window into how the hardware actually executes your code, enabling optimizations that would be difficult or impossible at higher levels of abstraction.

Whether you choose to write PTX directly, use inline PTX for specific operations, or simply inspect PTX to understand compiler behavior, the knowledge you've gained from this tutorial will help you write more efficient GPU code and make the most of NVIDIA's powerful hardware.

\bibliographystyle{plain}
\begin{thebibliography}{9}

\bibitem{nvidia_ptx_blog} NVIDIA Developer Blog, "Understanding PTX, the Assembly Language of CUDA GPU Computing," NVIDIA Technical Blog, 2020.

\bibitem{bruce_lee_tensor_core} Bruce Liang, "NVIDIA Tensor Core - Getting Started with MMA PTX Programming," Medium, 2022.

\bibitem{deepseek_ptx} Mark Craddock, "DeepSeek and DeepEP — Understanding DeepSeek's Custom CUDA PTX Instruction," Medium, 2025.

\bibitem{nvidia_ptx_forums} NVIDIA Developer Forums, "Problem generating .PTX files," NVIDIA Developer Forums, 2022.

\bibitem{nvidia_cuda_binary} NVIDIA, "CUDA Binary Utilities," NVIDIA Documentation, 2023.

\bibitem{stackoverflow_ptx_inline} Stack Overflow, "How to compile PTX code," Stack Overflow, 2013.

\bibitem{llvm_nvptx_guide} LLVM Project, "User Guide for NVPTX Back-end," LLVM Documentation, 2023.

\bibitem{arrayfire_ptx} ArrayFire, "Demystifying PTX Code," ArrayFire Blog, 2019.

\bibitem{cornell_gpu_memory} Cornell Virtual Workshop, "Understanding GPU Architecture - GPU Memory," Cornell University, 2020.

\end{thebibliography}

\end{document}