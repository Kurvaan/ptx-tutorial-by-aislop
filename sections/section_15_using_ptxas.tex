\subsubsection{Using ptxas}

To compile PTX to a cubin (CUDA binary), use:

\begin{lstlisting}[language=bash]
ptxas -arch=sm_80 -O3 myKernel.ptx -o myKernel.cubin
\end{lstlisting}

This command assembles the PTX file for compute capability 8.0 (Ampere) with optimization level 3. The output is a binary file that can be loaded by the CUDA driver.

Adding the \texttt{-v} (verbose) flag provides useful information about resource usage:

\begin{lstlisting}[language=bash]
ptxas -v -arch=sm_80 myKernel.ptx -o myKernel.cubin
\end{lstlisting}

This might output something like:
\begin{lstlisting}
ptxas info : Used 32 registers, 0 bytes spill stores, 0 bytes spill loads, 48 bytes shared memory, 136 bytes cmem[0]
\end{lstlisting}

This information is valuable for performance tuning as it shows how many registers per thread your kernel uses, whether any register spills occurred (which hurt performance), and shared memory usage.

\opttip{
Always watch for register spills in the \texttt{ptxas} output. If you see "spill stores/loads" > 0, it means the kernel ran out of registers and had to use slower local memory. Try to restructure your code to reduce register pressure if this happens.
}

