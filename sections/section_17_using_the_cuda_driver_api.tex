\subsubsection{Using the CUDA Driver API}

The CUDA Driver API allows loading PTX or cubin files at runtime and launching the kernels within them:

\begin{lstlisting}[language=C++]
#include <cuda.h>

int main() {
    // Initialize CUDA
    cuInit(0);
    CUdevice device;
    CUcontext context;
    cuDeviceGet(&device, 0);
    cuCtxCreate(&context, 0, device);
    
    // Load the PTX module
    CUmodule module;
    cuModuleLoad(&module, "myKernel.cubin");  // or load PTX directly
    
    // Get function handle
    CUfunction kernel;
    cuModuleGetFunction(&kernel, module, "myKernel");
    
    // Set up kernel parameters
    void* args[] = { &d_input, &d_output, &size };
    
    // Launch the kernel
    cuLaunchKernel(
        kernel,          // kernel function
        gridDimX, gridDimY, gridDimZ,  // grid dimensions
        blockDimX, blockDimY, blockDimZ,  // block dimensions
        sharedMemBytes,  // shared memory size
        stream,          // CUDA stream
        args,            // kernel arguments
        NULL             // extra options
    );
    
    // Clean up
    cuModuleUnload(module);
    cuCtxDestroy(context);
    
    return 0;
}
\end{lstlisting}

This approach gives you complete control over loading and executing PTX code, making it suitable for applications that dynamically select or generate kernels \citep{llvm_nvptx_guide}.

