\subsubsection{Kernel Entry Points}

Kernel functions—those that can be launched from host code—are declared with the \texttt{.entry} directive:

\begin{lstlisting}[style=ptx]
.visible .entry kernelName(
    .param .u64 param_A,    // first parameter (a pointer)
    .param .u32 param_N     // second parameter (an integer)
) {
    // kernel body
    // ...
    ret;
}
\end{lstlisting}

The components of this declaration are:

\begin{itemize}
    \item \texttt{.visible} - Makes the entry point externally visible, allowing the host to find it by name.
    
    \item \texttt{.entry} - Marks this as a kernel function (as opposed to a device function).
    
    \item \texttt{kernelName} - The name of the kernel, used when launching it from host code.
    
    \item \texttt{.param} declarations - List the parameters the kernel expects, with their types.
\end{itemize}

The parameters are declared with \texttt{.param} followed by a type qualifier (like \texttt{.u64} for a 64-bit unsigned integer or pointer) and a name. Inside the function body, these parameters are accessible via \texttt{ld.param} instructions that load from the parameter space into registers.

Device functions (callable only from device code) use \texttt{.func} instead of \texttt{.entry} and can omit the \texttt{.visible} qualifier if they're only used internally.

