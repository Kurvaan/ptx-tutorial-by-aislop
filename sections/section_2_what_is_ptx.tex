\subsection{What is PTX?}

PTX can be thought of as a virtual GPU assembly language—an intermediate representation that sits between your high-level CUDA code and the actual hardware instructions. When you compile a CUDA C/C++ program, the NVIDIA compiler (NVCC) first translates your device code into PTX, and then the PTX assembler (\texttt{ptxas}) converts that PTX into SASS (Streaming Assembly), which is the actual machine code for the specific GPU architecture \citep{nvidia_ptx_blog}.

\begin{figure}[h]
    \centering
    \begin{tikzpicture}[node distance=1.5cm, thick, rounded corners=2pt, 
                        every node/.style={font=\sffamily\small}]
        % Nodes with simpler styling
        \node (cudac) [draw, rectangle, minimum width=2.5cm, minimum height=0.8cm, 
                      fill=blue!20, text=blue!70!black] {CUDA C/C++};
        \node (ptx) [draw, rectangle, minimum width=2.5cm, minimum height=0.8cm, 
                    fill=green!20, text=green!70!black, right=of cudac] {PTX};
        \node (sass) [draw, rectangle, minimum width=2.5cm, minimum height=0.8cm, 
                     fill=red!20, text=red!70!black, right=of ptx] {SASS};
        \node (gpu) [draw, rectangle, minimum width=2.5cm, minimum height=0.8cm, 
                    fill=orange!20, text=orange!70!black, right=of sass] {GPU Execution};
        
        % Arrows with simpler labels
        \draw[->] (cudac) -- node[above, text=black, font=\scriptsize\ttfamily] {NVCC} (ptx);
        \draw[->] (ptx) -- node[above, text=black, font=\scriptsize\ttfamily] {ptxas} (sass);
        \draw[->] (sass) -- node[above, text=black, font=\scriptsize\ttfamily] {driver} (gpu);
    \end{tikzpicture}
    \caption{CUDA compilation pipeline showing the transformation from source code to GPU execution}
    \label{fig:cuda_pipeline}
\end{figure}

This design provides several advantages:

\begin{itemize}
    \item \textbf{Forward Compatibility}: PTX serves as a stable programming target that can be JIT-compiled by the GPU driver to run on newer hardware that didn't exist when the application was built \citep{nvidia_ptx_blog}.
    
    \item \textbf{Abstraction Layer}: PTX abstracts some hardware details while still providing low-level control over parallelism, memory operations, and synchronization.
    
    \item \textbf{Optimization Opportunities}: By inspecting or writing PTX directly, developers can implement optimizations that might be missed by the high-level compiler.
\end{itemize}

PTX is sometimes described as "CUDA's IR" (Intermediate Representation) \citep{bruce_lee_tensor_core}. Unlike pure machine code, PTX is somewhat higher-level—it supports symbolic registers, labels, and abstracts many hardware details. For example, a single PTX \texttt{add.f32} instruction may map to different actual instructions depending on the GPU generation.

