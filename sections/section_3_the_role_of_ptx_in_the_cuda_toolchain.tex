\subsection{The Role of PTX in the CUDA Toolchain}

The CUDA compilation process involves several steps, as illustrated in Figure \ref{fig:cuda_pipeline}:

\begin{enumerate}
    \item The CUDA compiler (NVCC) separates host (CPU) code and device (GPU) code
    \item The device code is compiled to PTX (an intermediate representation)
    \item The PTX code is then assembled by \texttt{ptxas} into binary cubin code for a specific target GPU
    \item Both PTX and cubin can be packaged into the final executable
\end{enumerate}

At runtime, the CUDA driver can:
\begin{itemize}
    \item Use the pre-compiled cubin if it matches the current GPU architecture
    \item Or just-in-time compile the embedded PTX if running on a newer GPU architecture
\end{itemize}

This approach ensures both performance (through pre-compiled binaries) and forward compatibility (through JIT compilation of PTX) \citep{nvidia_ptx_blog}.

\technote{
PTX is human-readable text resembling assembly language. This makes it possible to inspect, modify, or even write PTX code from scratch, giving developers another level of control over GPU execution.
}

