\subsubsection{Tile Loop Implementation}

Now we implement the loop over tiles. For each tile, we:
1. Load a portion of A and B from global to shared memory
2. Synchronize threads
3. Compute the partial dot product
4. Synchronize again before the next iteration

Here's a simplified version of the loop:

\begin{lstlisting}[style=ptx]
    // Initialize tile counter
    mov.u32 %r_t, 0;
    
TILE_LOOP_START:
    // Compute the indices for this tile
    // ... (index calculation for A and B)
    
    // Load A element from global to shared memory
    // Check bounds: if row < N && t*16 + threadIdx.x < N
    setp.lt.u32 %p0, %r_row, %r_N;  // row < N
    // ... (calculate index for t*16 + threadIdx.x)
    setp.lt.u32 %p1, %r_k_idx, %r_N;  // t*16 + threadIdx.x < N
    and.pred %p0, %p0, %p1;  // Both conditions true
    
    @!%p0 bra LOAD_A_ZERO;  // If not in bounds, load 0
    
    // Calculate global memory address for A[row, t*16 + threadIdx.x]
    // ... (address calculation)
    ld.global.f32 %f_a, [%rd_Aaddr];
    bra STORE_A_SHARED;
    
LOAD_A_ZERO:
    mov.f32 %f_a, 0f00000000;  // Load 0 if out of bounds
    
STORE_A_SHARED:
    // Calculate shared memory index shmemA[threadIdx.y][threadIdx.x]
    // ... (shared memory index calculation)
    st.shared.f32 [%rd_shAddrA], %f_a;
    
    // Similar logic for loading B into shared memory
    // ... (load B element)
    
    // Synchronize threads
    bar.sync 0;
    
    // Compute partial dot product for this tile
    // Initialize loop counter for k
    mov.u32 %r_k, 0;
    
DOT_PRODUCT_LOOP:
    // Calculate shared memory indices for A[threadIdx.y][k] and B[k][threadIdx.x]
    // ... (index calculations)
    
    // Load from shared memory
    ld.shared.f32 %f_a, [%rd_shA_k];
    ld.shared.f32 %f_b, [%rd_shB_k];
    
    // Multiply-accumulate: f_acc += f_a * f_b
    fma.rn.f32 %f_acc, %f_a, %f_b, %f_acc;
    
    // Increment k and check loop condition
    add.u32 %r_k, %r_k, 1;
    setp.lt.u32 %p0, %r_k, 16;  // k < 16
    @%p0 bra DOT_PRODUCT_LOOP;  // Jump back if k < 16
    
    // Synchronize before next tile
    bar.sync 0;
    
    // Increment tile counter and check if done
    add.u32 %r_t, %r_t, 1;
    // Calculate ceiling of N/16
    add.u32 %r_temp, %r_N, 15;
    div.u32 %r_num_tiles, %r_temp, 16;
    setp.lt.u32 %p0, %r_t, %r_num_tiles;  // t < ceiling(N/16)
    @%p0 bra TILE_LOOP_START;
    
    // Store final result to global memory if in bounds
    setp.lt.u32 %p0, %r_row, %r_N;  // row < N
    setp.lt.u32 %p1, %r_col, %r_N;  // col < N
    and.pred %p0, %p0, %p1;  // Both conditions true
    @!%p0 bra EXIT;  // Skip if out of bounds
    
    // Calculate global memory address for C[row, col]
    // ... (address calculation)
    st.global.f32 [%rd_Caddr], %f_acc;
    
EXIT:
    ret;
}
\end{lstlisting}

This PTX implementation is more verbose than the CUDA C++ version, but it illustrates how we explicitly manage:

\begin{itemize}
    \item \textbf{Thread indexing}: Manually computing row and column indices from block and thread IDs
    \item \textbf{Memory management}: Explicit loads and stores between global and shared memory
    \item \textbf{Control flow}: Using predicates and branches to implement loops and conditionals
    \item \textbf{Synchronization}: Explicit barriers to coordinate threads
\end{itemize}

