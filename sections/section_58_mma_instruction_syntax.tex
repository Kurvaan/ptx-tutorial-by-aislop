\subsubsection{MMA Instruction Syntax}

A typical \texttt{mma.sync} instruction might look like:

\begin{lstlisting}[style=ptx]
mma.sync.aligned.m16n8k16.row.col.f16.f16.f16.f16 
    {%d0,%d1,%d2,%d3}, 
    {%a0,%a1,%a2,%a3}, 
    {%b0,%b1}, 
    {%c0,%c1,%c2,%c3};
\end{lstlisting}

Let's break down the components of this instruction:

\begin{itemize}
    \item \texttt{mma.sync} - The base operation (matrix multiply-accumulate with warp synchronization)
    
    \item \texttt{aligned} - Memory alignment requirement
    
    \item \texttt{m16n8k16} - Shape of the matrix operation:
    \begin{itemize}
        \item \texttt{m16} - 16 rows in the output matrix (and in matrix A)
        \item \texttt{n8} - 8 columns in the output matrix (and in matrix B)
        \item \texttt{k16} - Contraction dimension (columns of A, rows of B)
    \end{itemize}
    
    \item \texttt{row.col} - Layout of matrices A and B (A is row-major, B is column-major)
    
    \item \texttt{f16.f16.f16.f16} - Data types for A, B, C, and D matrices (all FP16 in this case)
    
    \item The register lists in curly braces are the matrices' "fragments":

    \begin{itemize}
        \item \texttt{\{\%d0,\%d1,\%d2,\%d3\}} - Output matrix D (result of A×B+C)
        \item \texttt{\{\%a0,\%a1,\%a2,\%a3\}} - Input matrix A (16×16 in this example)
        \item \texttt{\{\%b0,\%b1\}} - Input matrix B (16×8 in this example)
        \item \texttt{\{\%c0,\%c1,\%c2,\%c3\}} - Accumulator matrix C (16×8, same dimensions as output D)
    \end{itemize}
\end{itemize}

The exact number of registers needed for each fragment depends on the matrix size and data type.

