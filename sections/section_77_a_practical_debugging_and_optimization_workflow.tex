\subsection{A Practical Debugging and Optimization Workflow}

Based on the tools described above, here's a recommended workflow for developing and optimizing PTX code:

\begin{enumerate}
    \item \textbf{Initial Development}:
    \begin{itemize}
        \item Write a reference implementation in CUDA C++ to establish correctness
        \item Generate PTX from the CUDA code to use as a starting point
        \item Modify the PTX with your optimizations
    \end{itemize}
    
    \item \textbf{Compilation and Basic Testing}:
    \begin{itemize}
        \item Compile with `ptxas -v` to check resource usage and catch errors
        \item Write a test harness to launch the kernel and validate results
        \item Compare output against the reference implementation
    \end{itemize}
    
    \item \textbf{Correctness Validation}:
    \begin{itemize}
        \item Use cuda-memcheck to detect memory errors
        \item Test edge cases and boundary conditions
        \item Run with small inputs where you can manually verify results
    \end{itemize}
    
    \item \textbf{Performance Analysis}:
    \begin{itemize}
        \item Profile with Nsight Compute to identify bottlenecks
        \item Examine the Source/PTX/SASS view to see which instructions cause stalls
        \item Check key metrics like achieved occupancy, memory throughput, and instruction mix
    \end{itemize}
    
    \item \textbf{Optimization Iteration}:
    \begin{itemize}
        \item Modify the PTX based on profiling insights
        \item Recompile and verify correctness is maintained
        \item Profile again to confirm improvements
    \end{itemize}
    
    \item \textbf{System Integration}:
    \begin{itemize}
        \item Integrate the optimized kernel into your application
        \item Use Nsight Systems to ensure it works well in the full system context
        \item Test with realistic workloads and data sizes
    \end{itemize}
\end{enumerate}

\opttip{
When optimizing PTX, focus on one issue at a time and measure the impact of each change. Performance tuning is often counterintuitive—what seems like an improvement might have unexpected effects due to complex interactions in the GPU pipeline.
}

